\documentclass{article}
\usepackage[utf8]{inputenc}
\usepackage{graphicx}
\usepackage{amsmath}
\usepackage{amssymb}
\usepackage{setspace}
\title{Summary of LSM tree_Survey}
\date{July 2019}
\begin{center}
\huge
\textbf {Summary of LSM Tree Survey }
\\
\end{center}
\begin{spacing}{5.5}
\end{spacing}
\begin{figure}[h]
\centering
\includegraphics[width=0.2\textwidth]{images/buet_logo_1.png}
\end{figure}
\begin{spacing}{5.5}
\end{spacing}
\begin{center}
\Large
Prepared By\\
MD.Omer Danish\\
Student ID: 1505053
\end{center}
\begin{spacing}{2.5}
\end{spacing}
\begin{center}
\Large
Submitted to \\
Md. Tarikul Islam Papon \\ \\ \\
Lecturer\\  \\ 
 \begin{spacing}{2.5}
\end{spacing}
\Large
\\
Department of Computer Science \\ \\
Bangladesh University of Engineering and Technology\\ \\ \\
\end{center}
\newpage

\begin{document}
Most of the modern systems are using 
The Log-Structured Merge-tree (LSM) in the storage layer to a great extent. Authors of this paper  provide a general taxonomy to classify the literature of LSM-trees and discuss their strengths and trade-offs.This design of LSM tree brings a number of advantages.Authors of this paper   survey these recent research efforts on improving LSM-trees. They also survey  some representative LSM-based NoSQL systems. 
\\ \\
\par
\newline
This paper contains mainly four topics.\textbf{Firstly} this paper  is reviewing  history of LSM-trees and  LSM-tree implementations at present.\textbf{ Secondly}    it 
presents a taxonomy of the proposed LSM-tree improvements
and surveys the existing work using that taxonomy.
\textbf{ Thirdly }it  surveys some representative LSM-based NoSQL
systems, focusing on their storage layers.\textbf{ Fourthly} it  reflects
on the result of this survey, identifying several outages and
opportunities for future work on LSM-based storage systems.
\\
\par
LSM tree maintains out-of-place update structure.This design improves write
performance and  can also simplify the recovery process by not overwriting
old data.The major problem of this design
is that read performance is sacrificed since a record may be
stored in any of multiple locations.
The LSM-tree addressed the  problems of log structured tree by designing a merge process which is integrated
into the structure itself, providing high write performance
with bounded query performance and space utilization.
\\
\par
Today’s LSM-tree implementations commonly exploit
the immutability of disk components to simplify concurrency
control and recovery. 
Leveling merge policy of LSM Tree optimizes for query performance
since there are fewer components to search in the LSM-tree.
The tiering merge policy is more write optimized since it reduces
the merge frequency.There are two well-known optimizations that are used by
most LSM-tree implementations today.One is Bloom filter and second is Partioning.Partioning is to range-partition the disk components of LSM-trees into
multiple (usually fixed-size) small partitions.The LSM-tree is highly tunable. For example, by changing
the merge policy from leveling to tiering, one can greatly
improve write performance with only a small negative impact on point lookup queries due to the Bloom filter

\par
\newline
\\
\\
%%3.1
Despite the popularity of LSM-trees
the basic LSM-tree design suffers from various draw backs and insufficiencie.
These are Write Amplification,Merge Operations,Hardware,Special Workloads,Auto-Tuning.
Secondary Indexing.Based on these major issues of the basic LSM-tree design,authors of the paper present a taxonomy of LSM-tree improvements.One Way to optimize write amplification is tiering with horizontal or al grouping.Another way to optimize  write performance skip-tree that proposes a merge skipping idea or Exploiting Data Skew.Write amplification for skewed update
workloads are reduced  where some hot keys are updated frequently.
The basic idea is to separate hot keys from cold keys in the memory
component so that only cold keys are flushed to disk.
\par
\newline
\\
\\
%%3.3
The paper reviews some existing work that improves the implementation
of merge operations, including improving merge performance, minimizing buffer cache misses, and eliminating write stalls.

\par 
\\
\\
%%3.4
This paper  reviews the LSM-tree improvements proposed for
different hardware platforms, including large memory, multi
core,SSD/NVM, and native storage.
\\
\\
%%3.5
This reviews LSM-tree improvements that
target special workloads like  temporal data, small
data, semi-sorted data, and append-mostly data to achieve better performance.
The improvements(LSM-trie,SlimDB ) presented in the paper each target a specialized workload.
\\
\\
%%3.6
The authors of this paper reviews some research task that are done to provide auto tuning of LSM trees.These works are Auto-Tuning,Parameter Tuning,Tuning Merge Policies,Dynamic Bloom Filter Memory Allocation,Optimizing Data Placement.
\\
%%3.7
The authors have discussed LSM-based secondary indexing techniques to
support efficient query processing, including index structures,index maintenance, statistics collection, and distributed indexing.These indexing  techniques are Index Structures,Index Maintenance,Statistics Collection,Distributed Indexing .
\\
\par
\newline
Authors also survey five representative LSM-based open
source NoSQL systems namely LevelDB, RocksDB ,
Cassandra, HBase, and AsterixDB.
LevelDB  is an LSM-based key-value store that  supports a simple key-value
interface including puts, gets, and scans.
Apache HBase  is a distributed data storage system in the
Hadoop ecosystem that is based on a master-slave architecture.
Apache AsterixDB is an open-source Big Data Management System (BDMS) that aims to manage massive amounts
of semi-structured data efficiently.
\\
\par
\newline

At the end of the paper the authors briefly discusses some future research directions
what they get from  the results of their survey.The fields are Thorough Performance Evaluation,Partitioned Tiering Structure,Hybrid Merge Policy,Minimizing Performance Variance,Towards Database Storage Engines.The authors presented everything systematically.Reading this survey paper has been a great experience. 


\end{document}
