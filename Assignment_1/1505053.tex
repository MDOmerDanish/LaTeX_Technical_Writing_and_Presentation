\documentclass{article}
\usepackage[utf8]{inputenc}
\usepackage{graphicx}
\usepackage{amsmath}
\usepackage{amssymb}


\begin{center}
 
\Large
\maketitle{CSE300\_Assignment1\\
Introduction to \LaTeX \\
\huge
Quantum Mechanics\\ \\ \\
\linebreak
\large
\author{MD.Omer Danish\\
Student ID: 1505053}
    
    
\end{center}


\vspace{5cm}
\hspace{4cm}
\includegraphics[width= .2\textwidth]{album/buet.png}\newline
\begin{center}
 Department of Computer Science and Engineering\\
 Bangladesh University of Engineering and Technology\\
(BUET)\\
Dhaka 1000\\
\date{\today}
\end{center}
\newpage
\tableofcontents

\newpage

\begin{document}



\section{Introduction}

\textbf{Quantum mechanics} (QM; also known as quantum physics or quantum theory), including quantum field theory, is a fundamental theory in physics which describes nature at the smallest scales of energy levels of atoms and subatomic particles.

\begin{figure}
    \centering
\includegraphics[width=\textwidth]{album/quantam.png}
    \captioig: Wave equation
\end{figure}

Classical physics (the physics existing before quantum mechanics) is a set of fundamental theories which describes nature at ordinary (macroscopic) scale. Most theories in classical physics can be derived from quantum mechanics as an approximation valid at large (macroscopic) scale.[3] Quantum mechanics differs from classical physics in that: energy, momentum and other quantities of a system may be restricted to discrete values (quantization), objects have characteristics of both particles and waves (wave-particle duality), and there are limits to the precision with which quantities can be known (uncertainty principle).



\section{History}


Scientific inquiry into the wave nature of light began in the 17th and 18th centuries, when scientists such as Robert Hooke, Christiaan Huygens and Leonhard Euler proposed a wave theory of light based on experimental observations.[7] In 1803, Thomas Young, an English polymath, performed the famous double-slit experiment that he later described in a paper titled On the nature of light and colours. This experiment played a major role in the general acceptance of the wave theory of light.

In 1838, Michael Faraday discovered cathode rays. These studies were followed by the 1859 statement of the black-body radiation problem by Gustav Kirchhoff, the 1877 suggestion by Ludwig Boltzmann that the energy states of a physical system can be discrete, and the 1900 quantum hypothesis of Max Planck.[8] Planck's hypothesis that energy is radiated and absorbed in discrete "quanta" (or energy packets) precisely matched the observed patterns of black-body radiation.


\section{Interactions with other scientific theories}
The rules of quantum mechanics are fundamental. They assert that the state space of a system is a Hilbert space (crucially, that the space has an inner product) and that observables of that system are Hermitian operators acting on vectors in that space—although they do not tell us which Hilbert space or which operators. These can be chosen appropriately in order to obtain a quantitative description of a quantum system.
\subsection{Quantum mechanics and classical physics}

Predictions of quantum mechanics have been verified experimentally to an extremely high degree of accuracy.[46] According to the correspondence principle between classical and quantum mechanics, all objects obey the laws of quantum mechanics, and classical mechanics is just an approximation for large systems of objects (or a statistical quantum mechanics of a large collection of particles).
\subsection{Copenhagen interpretation of quantum versus classical kinematics}
A big difference between classical and quantum mechanics is that they use very different kinematic descriptions.[54]

In Niels Bohr's mature view, quantum mechanical phenomena are required to be experiments, with complete descriptions of all the devices for the system, preparative, intermediary, and finally measuring. The descriptions are in macroscopic terms, expressed in ordinary language, supplemented with the concepts of classical mechanics.[55][56][57][58] The initial condition and the final condition of the system are respectively described by values in a configuration space, for example a position space, or some equivalent space such as a momentum space. 

\subsection{General relativity and quantum mechanics}
Even with the defining postulates of both Einstein's theory of general relativity and quantum theory being indisputably supported by rigorous and repeated empirical evidence, and while they do not directly contradict each other theoretically (at least with regard to their primary claims), they have proven extremely difficult to incorporate into one consistent, cohesive model.[68]

\subsection{Attempts at a unified field theory}
The quest to unify the fundamental forces through quantum mechanics is still ongoing. Quantum electrodynamics (or "quantum electromagnetism"), which is currently (in the perturbative regime at least) the most accurately tested physical theory in competition with general relativity,[70][71] has been successfully merged with the weak nuclear force into the electroweak force and work is currently being done to merge the electroweak and strong force into the electrostrong force. 

\section{Schrodinger Equation}

In quantum mechanics, the Schrödinger equation is a mathematical equation that describes the changes over time of a physical system in which quantum effects, such as wave–particle duality, are significant. The equation is a mathematical formulation for studying quantum mechanical systems. It was named after Erwin Schrödinger, who derived the equation in 1925 and published it in 1926, forming the basis for his work that resulted in his being awarded the Nobel Prize in Physics in 1933.[1][2] The equation is a type of differential equation known as a wave-equation, which serves as a mathematical model of the movement of waves.
the uqn is given below
\begin{equation}
    
i\hslash \dfrac{\partial}{\partial t}|\Psi(t)>=\Hat{H}|\Psi(t)>

\end{equation}


\section{Application}
 Some of the applications where this theory is used  are given below 

\begin{itemize}
    \item Electronics
    \item Cryptography
    \item Quantum computing
    \item Quantum theory
    \item Macroscale quantum effects
\end{itemize}

\section{Examples}
Thhese are the following cases where we need quantum theory in a great content
\begin{enumerate}
\item Free particle
\item Particle in a box
\item Finite potential well
\item Rectangular potential barrier
\item Harmonic oscillator
\item Step potential
   
\end{enumerate}

\section{Conclusion}
This   \textbf{Quantum mechanics}  open up new door for the world.We should make the best use of it for mankind.

\end{document}
