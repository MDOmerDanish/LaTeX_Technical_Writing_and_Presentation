\documentclass{beamer}
\usepackage[utf8]{inputenc}
\usepackage{subfiles}
\usepackage{tikz}
\usepackage{color}

\usetheme{Madrid}
\usecolortheme{beaver}

\title[Being Greedy]{Greedy Algorithm}
%% \subtitle{in sleepy mode}

\author[Sagor and Danish]{Sagor Biswas  \newline Md. Omer Danish}

\institute[BUET]
{

Department of Computer Science and Engineering \\
BUET\\
}
\date{\today}


\AtBeginSection
{
\begin{frame}{We will Discuss}
    \tableofcontents[currentsection]
\end{frame}
}

\begin{document}
\titlepage

\begin{frame}{Contents}
    \tableofcontents
\end{frame}

\section{A real life scenario}
%\newline

\begin{frame}{A real life scenario}
There is a very rich man who has infinite supply of 500 Taka ,100 Taka  and 50 Taka notes.\pause
\newline
He has a friend who needs some money. \\He asks for 650 Taka but in minimum number of notes.


\end{frame}

\subfile{figures/image.tex}

\begin{frame}{Some possible ways to give money}

\newline
If there was no restriction, he could have given it in many ways.\pause \\Such as ...\\

\setbeamercovered{dynamic}
\begin{enumerate}
    \item <2->50+50+50+100+100+100+100+100
    \item<3-> 50+50+50+500 
    \item<4-> 50+50+50+50+50+100+100+100+100
    
    \item<5->  50+50+50+50+50+50+50+100+100+100 .
    



\end{enumerate}

\end{frame}







\begin{frame}{Best solution to give money}

But do they satisfy the condition of minimum note number?? \newline No...\\ \\ \pause
\newline

After thinking for a while he proposes a solution 
\newline  500+100+50\\ \pause
\newline  How did he come to the solution???\\ \\ \pause
\newline
For every iteration he has to give note of largest value that does not take him past the amount to be given.\pause
\newline  


\textbf{Actually  this  is  Greedy Algorithm}
\end{frame}





\section{Introduction to Greedy Algorithm}

\begin{frame}{Greedy Algorithm}
%\centering
A greedy algorithm is an algorithmic paradigm that follows the problem solving heuristic of making the locally optimal choice at each stage with the intent of finding a global optimum. 
\newline Let's see it's Pros and Cons \newline \pause
\newline 
\begin{itemize}
        \item Simple,easy to implement and runs fast\pause
        \item  But very often they don't provide a globally optimum.
    \end{itemize} 
\newline .
 


\end{frame}



\section{Problem with Greedy algorithm}

\begin{frame}{Problems with Greedy Approach }
Find a path from root to leaf having maximum sum\newline

%\only<1>{ Find the path having maximum sum\newline}
%\only<2>{ Found path with Greedy Algorithm\newline}
%\only<3->{ Actual Solution to the problem\newline}
%\setbeamercovered{dynamic}
\only<1> \subfile{figures/Graph.tex}
\only<2> \subfile{figures/GreedySolution.tex}
\only<3->\subfile{figures/RightSolution.tex} 

\end{frame}

\begin{frame}{Problem characteristics to use this algorithm }

Problems on which greedy approach work has two properties \newline
\setbeamercovered{dynamic}

\begin{enumerate}
    %\centering
    \item \textbf{1.Greedy-choice property}
    \begin{itemize}
        \item A global optimum can be arrived at by selecting a local optimum.
        
    \end{itemize} \pause
    
    \item \textbf{2.Optimum substructure property}
    \begin{itemize}
        \item An optimum solution to the problem contains an optimum solution to the subproblems.
        
    \end{itemize} 
\end{enumerate}

\end{frame}

\section{Applications}
\begin{frame}{Applications}
\setbeamercovered{dynamic}
\begin{enumerate}
    %\centering
    \item Huffman Coding
    \item<2-> Fractional Knapsack Problem 
    \item<3-> Prim's Minimum Spanning Tree
    
    \item<4-> Job Sequencing Problem
    
    \item<5-> Activity Selection Problem



\end{enumerate}

\end{frame}



\end{document}
